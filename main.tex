\documentclass[a4paper,ngerman]{report}

\usepackage{hyperref}

\usepackage{imakeidx}
\makeindex
\newcommand{\Index}[1]{\emph{#1\index{#1}}}

\usepackage{babel}

\usepackage{hyperref}
\usepackage{amsmath, amssymb, amsthm}
\usepackage{mathtools}

\usepackage{tikz}
\usetikzlibrary{automata,positioning,shapes.multipart}

\usepackage{thmtools}
\declaretheorem[name=Definition,style=definition,numberwithin=section]{definition}
\declaretheorem[name=Bemerkung,style=definition,sibling=definition]{remark}
\declaretheorem[name=Notation,style=definition,sibling=definition]{notation}
\declaretheorem[sibling=definition]{theorem}
\declaretheorem[name=Beispiel,style=definition,sibling=definition]{example}
\declaretheorem[sibling=definition]{lemma}

\usepackage{enumitem}

\usepackage{authorchapters/authorchapters}

\usepackage{bussproofs}

\newcommand{\Nat}{\mathbb{N}}
\newcommand{\Reel}{\mathbb{R}}

\title{Materialien zur Vorbereitung auf das 1.~Staatsexamen Informatik in Bayern}
\author{FSI Lehramt Informatik (Erlangen)}

\begin{document}
\maketitle
\chapter*{Lizenz}
Dieses Werk ist unter einer
\href{https://creativecommons.org/licenses/by-nc-sa/4.0/}{Creative Commons
Attribution-NonCommercial-ShareAlike 4.0 International License} lizenziert. Es
kann wie folgt referenziert werden:
\begin{quote}
  Materialien zur Vorbereitung auf das 1.~Staatsexamen Informatik in Bayern
  \copyright{} 2024 by FSI Lehramt Informatik (Erlangen) is licensed under CC BY-NC-SA 4.0.
  To view a copy of this license, visit \url{https://creativecommons.org/licenses/by-nc-sa/4.0}.
\end{quote}

\tableofcontents
\chapter{Vorwort}
\section{Aufbau und Motivation}
Ziel dieses Dokuments und des zugehörigen Repositorys ist es, eine studentische
Sammlung an Lösungsvorschlägen zu den Aufgaben vergangener 1. Staatsexamen im
Fach Informatik in Bayern zu erstellen. Wir wollen dadurch kooperative und
nachhaltige Vorbereitung fördern. Da wir leider kein Urheberrecht an den
Aufgabenstellungen haben, können wir diese jedoch nicht veröffentlichen. Wir
bitten um Verständnis.

Als Ausgleich dafür soll diese Sammlung auch einen Überblick über die
Themenbereiche der Informatik geben, die in den Staatsexamen abgefragt werden.
Dabei soll idealerweise eine Sammlung an Definitionen, Algorithmen, Sätzen und
Ähnlichem entstehen, die potenziell mit eigens erstellten Beispielen ergänzt
werden.

\sectionwithauthor[author={Max Mustermann},email={max@mustermann.de}]{Mitmachen}
Wir freuen uns selbstverständlich über jede Art von Beitrag. Sei es das Hinzufügen von Lösungen,
das Korrigieren von Fehlern oder das Ergänzen von Inhalten in den
Themenbereichen. Ausführliche Informationen zum Mitwirken lassen sich auf
\href{https://github.com/fsi-la-inf/stex-stuff/.github/CONTRIBUTING.md}{Github} finden.

Wir möchten auch die Möglichkeit geben, eigene Lösungsvorschläge für
Aufgaben mit den eigenen Autor-Informationen zu versehen. Dies sieht man
beispielsweise an diesem Unterkapitel. Dies ist aber komplett optional!

\chapter{Themenbereiche}

\chapter{Lösungsversuche}


\printindex{}
\end{document}
