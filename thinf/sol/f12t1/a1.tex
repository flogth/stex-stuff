\subsubsectionwithauthor[author={Max Ole Elliger},email={ole.elliger@fau.de}]{Aufgabe 1}

Definiere zunächst
\begin{align*}
  A_1 & \coloneqq \{ (i,t) \in \Nat \times \Nat \mid M_i\ \text{hält auf Eingabe}\ i\ \text{nicht innerhalb von}\ t\ \text{Schritten} \} \\
  A_2 & \coloneqq \{ (i,x) \in \Nat \times \Nat \mid M_i\ \text{hält nicht auf die Eingabe}\ x\} \\
  A_3 & \coloneqq \{ i \in \Nat \mid M_i\ \text{berechnet}\ f
        \colon \Nat \to \Nat\ \text{und}\ \exists x \in
        \Nat .\, f(x) = x \}
\end{align*}

\paragraph{$A_1$ ist entscheidbar.}
Folgende TM $M$ entscheidet $A_1$:
\begin{enumerate}
	\item
		Gegeben $(i,t) \in \Nat \times \Nat$.
	\item
		Simuliere $M_i$ auf Eingabe $i$.
	\item
		Wenn $M_i$ mindestens $t+1$ Schritte rechnet, halte und
		akzeptiere. Ansonsten halte und akzeptiere nicht.
\end{enumerate}
\par

\paragraph{$A_2$ ist co-r.e..}
$A_2$ ist das Komplement des Halteproblems. Da das Halteproblem
semi-entscheidbar (also r.e.) ist, kann $A_2$ nicht auch r.e. sein, ist also
co-r.e.
\par

\paragraph{$A_3$ ist r.e..}
Folgende NTM $N$ semi-entscheidet $A_3$:
\begin{enumerate}
	\item
		Gegeben $i \in \Nat$.
	\item
		Wähle nicht-deterministisch ein $x \in \Nat$.
	\item
		Simuliere $M_i$ mit Eingabe $x$.
	\item
		Falls das berechnete Ergebnis wieder $x$ ist: Halte und
		akzeptiere.
\end{enumerate}
\par

\paragraph{$A_3$ ist nicht entscheidbar.}
Es gilt:
\[
	A_3 = \{ i \in \Nat \mid M_i\ \text{berechnet}\ f \in \{f \mid
	\exists x \in \Nat .\, f(x) = x\} \}
\]
Definiere nun
\[
	S \coloneqq \{f \mid \exists x \in \Nat .\, f(x) = x\}
\]
Es gilt:
\begin{align*}
	(x \mapsto x) & \in S \\
	(x \mapsto x+1) & \notin S
\end{align*}
Mithilfe des Satzes von Rice folgt nun, dass die $A_3$ nicht entscheidbar ist.
\par
