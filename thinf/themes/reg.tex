\section{Reguläre Sprachen}

\begin{definition}[Reguläre Grammatik]
	Eine Grammatik $G = (V, \Sigma, P, S)$ heißt \Index{regulär}, alle durch
	$P$ induzierten Regeln von einer der folgenden Formen sind:
	\begin{align*}
		A & \rightarrow aB & A,B \in V, a \in \Sigma \\
		A & \rightarrow a & a \in \Sigma \\
		A & \rightarrow \epsilon
	\end{align*}
\end{definition}

\begin{definition}[Reguläre Sprache]
	Eine Sprache $L$ heißt \Index{regulär}, wenn es eine reguläre Grammatik
	$G$ gibt, sodass $L = L(G)$.
\end{definition}

\begin{theorem}[Pumping-Lemma]
	Sei $L$ eine formale Sprache.
	Wenn $L$ regulär ist, dann gibt es eine Zahl $n \geq 1$, sodass für
	jedes Wort $w \in L$ mit $|w| \geq n$ drei Teilwörter $x,y,z \in
	\Sigma^*$ existieren, sodass
	\begin{align}
		w & = xyz \tag{PL-Partition}\label{thinf:reg:pl-partition} \\
		|xy| & \leq n \label{thinf:reg:pl-beginning} \\
		y & \neq \epsilon \label{thinf:reg:pl-midnotempty} \\
		\forall p \in \Nat.\, xy^pz & \in L
		\tag{PL-Pumping}\label{thinf:reg:pl-pumping}
	\end{align}
\end{theorem}

\begin{example}
	\label{thinf:reg:pl-example-1}
	Die Sprache aus \eqref{thinf:cf:lang_1} ist nicht regulär. Dies lässt
	sich zum Beispiel durch einen Widerspruchsbeweis mittels Pumping-Lemma
	zeigen.

	Sei $n \geq 1$ die Pumping-Zahl. Es gilt
	\begin{align*}
		w & := a^n b c^n \in L \\
		|w| & = 2n + 1 > n
	\end{align*}

	Es gibt also $x,y,z \in \Sigma^*$, sodass
	\eqref{thinf:reg:pl-partition}, \eqref{thinf:reg:pl-beginning},
	\eqref{thinf:reg:pl-midnotempty} und \eqref{thinf:reg:pl-pumping}
	gelten.

	Wegen \eqref{thinf:reg:pl-beginning} folgt, dass ein $k \in \Nat$ existiert, sodass
	\begin{equation}
		\label{thinf:reg:pl-example-1:eq-1}
		y = a^k
	\end{equation}

	Wegen \eqref{thinf:reg:pl-midnotempty} und
	\eqref{thinf:reg:pl-example-1:eq-1} folgt direkt
	\begin{equation}
		\label{thinf:reg:pl-example-1:eq-2}
		k \geq 1
	\end{equation}

	Wegen \eqref{thinf:reg:pl-pumping} müsste nun auch gelten, dass
	\begin{equation}
		w' := xy^0z \in L
	\end{equation}

	Dies ist aber nicht der Fall, wie folgende Rechnung zeigt:
	\begin{align*}
		w'
		& = xy^0z \\
		& = xz \\
		& = a^{n - k} b c^n \\
	\end{align*}

	Aus \eqref{thinf:reg:pl-example-1:eq-2} folgt $n - k \neq n$, und damit ist $w' \notin L$, was im
	Widerspruch zur Aussage des Pumping-Lemmas steht. Folglich kann $L$ nicht
	regulär sein.
\end{example}
