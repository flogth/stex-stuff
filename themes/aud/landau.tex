\section{Landau-Notation}

\begin{definition}
	Sei $g \colon \Nat \rightarrow \Nat$ eine Funktion. Dann ist $O(g)$
	wie folgt definiert:
	\begin{equation}
		O(g) := 
		\left\{
			f \colon \Nat \rightarrow \Reel^+ \,|\,
			\begin{aligned}
				\exists n_0 \in \Nat.\, \\
				\exists c_2 > 0 \in \Reel.\, \\
				\forall n \geq n_0 \in \Nat.\, \\
				f(n) \leq c_2 \cdot g(n) \\
			\end{aligned}
		\right\}
	\end{equation}
\end{definition}

\begin{definition}
	Sei $g \colon \Nat \rightarrow \Nat$ eine Funktion. Dann ist $\Omega(g)$
	wie folgt definiert:
	\begin{equation}
		\Omega(g) := 
		\left\{
			f \colon \Nat \rightarrow \Reel^+ \,|\, 
			\begin{aligned}
			\exists n_0 \in \Nat.\, \\
			\exists c_1 > 0 \in \Reel.\, \\
			\forall n \geq n_0 \in \Nat.\, \\
			c_1 \cdot g(n) \leq f(n) \\
			\end{aligned}
		\right\}
	\end{equation}
\end{definition}

\begin{definition}
	Sei $g \colon \Nat \rightarrow \Nat$ eine Funktion. Dann ist $\Theta(g)$
	wie folgt definiert:
	\begin{equation}
		\Theta(g) := 
		\left\{
			f \colon \Nat \rightarrow \Reel^+ \,|\, 
			\begin{aligned}
			\exists n_0 \in \Nat.\, \\
			\exists c_1 > 0 \in \Reel.\, \\
			\exists c_2 > 0 \in \Reel.\, \\
			\forall n \geq n_0 \in \Nat.\, \\
			c_1 \cdot g(n) \leq f(n) \leq c_2 \cdot g(n) \\
			\end{aligned}
		\right\}
	\end{equation}
\end{definition}

%TODO Rechenregeln

\begin{example}
	\begin{align*}
		O(3^{log_2{n}})
		& = O(3^{log_3{n}}) \\
		& = O(n) \\
		& \subseteq O(n^2 - \log_2{n^{10}}) \\
		& = O(n^2 - 10 \cdot \log_2{n}) \\
		& = O(n^2) \\
		& = O(3n^2 + \sqrt{4n}) \\
		& \subseteq O(2^{n/2}) \\
		& \subseteq O(3^{n-1})
	\end{align*}
\end{example}

\begin{lemma}
	Für alle Funktionen $f,g,h \colon \Nat \rightarrow \Nat$ gilt,
	wenn $f \in \Theta(h)$ und $g \in O(h)$, dann auch $f + g \in
	\Theta(h)$.
\end{lemma}
\begin{proof}
	Seien $f,g,h \colon \Nat \rightarrow \Nat$ Funktionen, sodass gilt:
	\begin{align}
		f \in \Theta(h) \label{f_theta} \\
		g \in O(h) \label{g_O}
	\end{align}
	Zu zeigen ist:
	\begin{equation}
		f + g \in \Theta(h) \label{fg_theta}
	\end{equation}
	Aus \eqref{f_theta} folgt, dass $n_{f0},c_{f1},c_{f2}$ existieren, 
	sodass für alle $n \geq n_{f0}$ folgende Gleichung gilt:
	\begin{equation}
		c_{f1} \cdot h(n) \leq f(n) \leq c_{f2} \cdot h(n)
	\end{equation}
	Aus \eqref{g_O} folgt, dass $n_{g0},c_{g2}$ existieren, sodass für alle
	$n \geq n_{g0}$ folgende Gleichung gilt:
	\begin{equation}
		g(n) \leq c_{g2} \cdot h(n)
	\end{equation}
	Gleichung \eqref{fg_theta} bedeutet, dass $n_0,c_1,c_2$ existieren,
	sodass für alle $n \geq n_0$ gilt, dass
	\begin{equation}
		c_1 \cdot h(n) \leq f(n) + g(n) \leq c_2 \cdot h(n)
	\end{equation}
	Wähle nun $n_0 = \max{\{n_{f0},n_{g0}\}}$, $c_1 = c_{f1}$ und $c_2 = c_{f2} + c{g2}$. Dann
	gilt:
	\begin{align*}
		\bf c_1 \cdot h(n)
		& = c_{f1} \cdot h(n) \\
		& \leq f(n) \\
		& \leq \bf f(n) + g(n) \\
		& \leq c_{f2} \cdot h(n) + c_{g2} \cdot h(n) \\
		& = (c_{f2} + c_{g2}) \cdot h(n) \\
		& = \bf c_2 \cdot h(n)
	\end{align*}

	Dies war noch zu zeigen.
\end{proof}

\begin{lemma}
	Es gibt Funktionen $f,g,h \colon \Nat \rightarrow \Nat$, sodass
	\begin{align}
		f \in \Theta (h) \label{cex-1} \\
		g \in \Omega (h) \label{cex-2} \\
		f + g \notin \Theta (h) \label{cex-3}
	\end{align}
\end{lemma}
\begin{proof}
	Wähle $f(n) := 1$, $g(n) := 2^n$ und $h(n) := 1$.

	Dann folgt \eqref{cex-1} direkt, da $f = g$. \eqref{cex-2} folgt, weil
	für alle $n \geq 1$ gilt, dass $h(n) \leq g(n)$.

	Jedoch hat $f + g$ nicht durchschnittlich konstante Laufzeit.
	\eqref{cex-3} lässt sich zeigen, da für jedes $n_0,c_1,c_2 \in \Nat$
	(gemäß obiger Definition) ein $n \geq n_0$ existiert, sodass $f(n) +
	g(n) > c_2$, nämlich $n = 2^{c_2 + 1}$.
\end{proof}

\begin{theorem}[Master-Theorem]
	Sei folgende Rekursionsgleichung gegeben:
	\begin{equation}
		T(n) = a \cdot T(\frac{n}{b}) + f(n)
	\end{equation}

	Dann gelten folgende Implikationen:
	\begin{prooftree}
		\AxiomC{
			$
			\exists \epsilon > 0.\, 
			f \in O(n^{\log_b{a - \epsilon}})
			$
		}
		\UnaryInfC{
			$
			T \in \Theta(n^{\log_b{a}})
			$
		}
	\end{prooftree}
	\begin{prooftree}
		\AxiomC{
			$
			f(n) \in \Theta(n^{\log_b{a}})
			$
		}
		\UnaryInfC{
			$
			T \in \Theta(n^{\log_b{a}}\cdot\log{n})
			$
		}
	\end{prooftree}
	\begin{prooftree}
		\AxiomC{
			$
			\exists \epsilon > 0.\, 
			f \in \Omega(n^{\log_b{a+\epsilon}})
			$
		}
		\AxiomC{
			$
			\exists 0 < c < 1.\, 
			\exists n_0 \in \Nat.\, 
			\forall n \geq n_0.\, 
			a \cdot f(\frac{n}{b}) \leq c \cdot f(n)
			$
		}
		\BinaryInfC{
			$
			T \in \Theta(f)
			$
		}
	\end{prooftree}
\end{theorem}

\begin{example}
	Es ist folgende Gleichung gegeben:
	\begin{equation*}
		T(n) = 3 \cdot T(\frac{n}{5}) + \frac{n}{2}
	\end{equation*}
	Definiere
	\begin{align*}
		a & := 3 \\
		b & := 5 \\
		f(n) & := \frac{n}{2}
	\end{align*}
	Es gilt
	\[
		f \in \Omega(n^{\log_5{3+2}})
	\]
	sowie
	\begin{align*}
		3 \cdot f(\frac{n}{5})
		& = \frac{3n}{10} \\
		& = \frac{3}{5} \cdot \frac{n}{2} \\
		& \leq \frac{3}{5} \cdot \frac{n}{2} \\
		& \leq \frac{3}{5} \cdot f(n) \\
	\end{align*}
	Damit folgt nun nach Master-Theorem, dass $T \in \Theta(f)$.
\end{example}
